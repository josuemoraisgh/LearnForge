\documentclass[12pt]{beamer}
\usepackage{bookmark}
\usepackage[utf8]{inputenc}
\usepackage{amsmath}
\usepackage{amssymb}
\usetheme{Madrid}
\usepackage{graphicx}
\usepackage{enumitem}
\title{Exercicios de Teste}
\author{}
\date{}

\begin{document}

\frame{\titlepage}

\setbeamerfont{frametitle}{size=\small}

\newcommand{\BodySize}{\footnotesize}

\begin{frame}
\frametitle{1) Em sistemas de monitoramento térmico, a escolha do sensor adequado depende de fatores como precisão, necessidade de calibração, princípio de funcionamento e compatibilidade com circuitos eletrônicos. Considerando um cenário em que se deseja medir temperatura sem necessidade de calibração prévia, com resposta rápida, sem depender da variação resistiva e utilizando eletrônica apropriada, qual dos sensores abaixo atende a essas exigências?}
{\BodySize
\begin{itemize}
\item[a)] Um termistor PTC, cuja resistência aumenta com a temperatura, exigindo circuitos de calibração para evitar erros em faixas mais amplas.
\item[b)] Uma RTD (termoresistência), que apresenta alta precisão, mas requer excitação elétrica e calibração prévia para funcionamento adequado.
\item[c)] Um cristal piezoelétrico, que converte pressão mecânica em sinais elétricos, sendo mais indicado para medições dinâmicas do que térmicas.
\item[d)] Um termistor NTC, cuja resistência diminui com o aumento da temperatura, exigindo compensações eletrônicas para leituras lineares.
\item[e)] Um termopar, formado por dois metais distintos, que gera uma força eletromotriz proporcional à diferença de temperatura entre suas junções.
\end{itemize}
}
\end{frame}

\begin{frame}
\frametitle{1) Em sistemas de monitoramento térmico, a escolha do sensor adequado depende de fatores como precisão, necessidade de calibração, princípio de funcionamento e compatibilidade com circuitos eletrônicos. Considerando um cenário em que se deseja medir temperatura sem necessidade de calibração prévia, com resposta rápida, sem depender da variação resistiva e utilizando eletrônica apropriada, qual dos sensores abaixo atende a essas exigências?}
{\BodySize
\begin{itemize}
\item[a)] Um termistor PTC, cuja resistência aumenta com a temperatura, exigindo circuitos de calibração para evitar erros em faixas mais amplas.
\item[b)] Uma RTD (termoresistência), que apresenta alta precisão, mas requer excitação elétrica e calibração prévia para funcionamento adequado.
\item[c)] Um cristal piezoelétrico, que converte pressão mecânica em sinais elétricos, sendo mais indicado para medições dinâmicas do que térmicas.
\item[d)] Um termistor NTC, cuja resistência diminui com o aumento da temperatura, exigindo compensações eletrônicas para leituras lineares.
\item[e)] \alert{Um termopar, formado por dois metais distintos, que gera uma força eletromotriz proporcional à diferença de temperatura entre suas junções.}
\end{itemize}
}
\end{frame}

\begin{frame}
\frametitle{1) Em sistemas de monitoramento térmico, a escolha do sensor adequado depende de fatores como precisão, necessidade de calibração, princípio de funcionamento e compatibilidade com circuitos eletrônicos. Considerando um cenário em que se deseja medir temperatura sem necessidade de calibração prévia, com resposta rápida, sem depender da variação resistiva e utilizando eletrônica apropriada, qual dos sensores abaixo atende a essas exigências?}
{\BodySize
\textbf{OBS.:}
\begin{itemize}
\item Este é um exempo de observação
\item Veja como são as coisas
\end{itemize}
}
\end{frame}

\begin{frame}
\frametitle{2) A medida de profundidade em ambientes aquáticos está relacionada à pressão hidrostática, através da relação aproximadamente linear P = f(z), em que z é a profundidade e P é a pressão. Assuma que a densidade da água do mar ρ = 1,025 x 103 kg.m-3, que não há variação dessa densidade com a profundidade e que o valor da aceleração da gravidade é g = 9,8 m/s2. Nesse contexto, assinale a opção cujo gráfico relaciona adequadamente a profundidade com a pressão hidrostática:}
{\BodySize
\begin{itemize}
\item[a)] \fbox{\rule{0pt}{4cm}\rule{6cm}{0pt}}
\item[b)] \fbox{\rule{0pt}{4cm}\rule{6cm}{0pt}}
\item[c)] \fbox{\rule{0pt}{4cm}\rule{6cm}{0pt}}
\item[d)] \fbox{\rule{0pt}{4cm}\rule{6cm}{0pt}}
\item[e)] \fbox{\rule{0pt}{4cm}\rule{6cm}{0pt}}
\end{itemize}
}
\end{frame}

\begin{frame}
\frametitle{2) A medida de profundidade em ambientes aquáticos está relacionada à pressão hidrostática, através da relação aproximadamente linear P = f(z), em que z é a profundidade e P é a pressão. Assuma que a densidade da água do mar ρ = 1,025 x 103 kg.m-3, que não há variação dessa densidade com a profundidade e que o valor da aceleração da gravidade é g = 9,8 m/s2. Nesse contexto, assinale a opção cujo gráfico relaciona adequadamente a profundidade com a pressão hidrostática:}
{\BodySize
\begin{itemize}
\item[a)] \fbox{\rule{0pt}{4cm}\rule{6cm}{0pt}}
\item[b)] \fbox{\rule{0pt}{4cm}\rule{6cm}{0pt}}
\item[c)] \fbox{\rule{0pt}{4cm}\rule{6cm}{0pt}}
\item[d)] \fbox{\rule{0pt}{4cm}\rule{6cm}{0pt}}
\item[e)] \fbox{\rule{0pt}{4cm}\rule{6cm}{0pt}}
\end{itemize}
}
\end{frame}

\begin{frame}
\frametitle{2) A medida de profundidade em ambientes aquáticos está relacionada à pressão hidrostática, através da relação aproximadamente linear P = f(z), em que z é a profundidade e P é a pressão. Assuma que a densidade da água do mar ρ = 1,025 x 103 kg.m-3, que não há variação dessa densidade com a profundidade e que o valor da aceleração da gravidade é g = 9,8 m/s2. Nesse contexto, assinale a opção cujo gráfico relaciona adequadamente a profundidade com a pressão hidrostática:}
{\BodySize
\textbf{OBS.:}
\begin{itemize}
\item Este é um exempo de observação
\item Veja como são as coisas
\end{itemize}
}
\end{frame}

\begin{frame}
\frametitle{3) Um sensor RTD de platina apresenta resistência nominal de <R0>Ω a 0 °C. Sabendo que a resistência medida foi de <RM>Ω e que o coeficiente de temperatura do sensor é <ALPHA> 1/°C, calcule a temperatura do sensor considerando a equação:

RF = RI × (1 + α × ΔT)

Qual é o valor de ΔT (temperatura atual)?}
{\BodySize
\begin{itemize}
\item[a)] O valor correto é <TEMP + 1>
\item[b)] O valor correto é <TEMP / 1.5>
\item[c)] O valor correto é <TEMP - 2.5>
\item[d)] O valor correto é <TEMP * 0.5>
\item[e)] O valor correto é <TEMP>
\end{itemize}
}
\end{frame}

\begin{frame}
\frametitle{3) Um sensor RTD de platina apresenta resistência nominal de <R0>Ω a 0 °C. Sabendo que a resistência medida foi de <RM>Ω e que o coeficiente de temperatura do sensor é <ALPHA> 1/°C, calcule a temperatura do sensor considerando a equação:

RF = RI × (1 + α × ΔT)

Qual é o valor de ΔT (temperatura atual)?}
{\BodySize
\begin{itemize}
\item[a)] O valor correto é <TEMP + 1>
\item[b)] O valor correto é <TEMP / 1.5>
\item[c)] O valor correto é <TEMP - 2.5>
\item[d)] O valor correto é <TEMP * 0.5>
\item[e)] \alert{O valor correto é <TEMP>}
\end{itemize}
}
\end{frame}

\begin{frame}
\frametitle{3) Um sensor RTD de platina apresenta resistência nominal de <R0>Ω a 0 °C. Sabendo que a resistência medida foi de <RM>Ω e que o coeficiente de temperatura do sensor é <ALPHA> 1/°C, calcule a temperatura do sensor considerando a equação:

RF = RI × (1 + α × ΔT)

Qual é o valor de ΔT (temperatura atual)?}
{\BodySize
\textbf{OBS.:}
\begin{itemize}
\item Este é um exempo de observação
\item Veja como são as coisas
\end{itemize}
}
\end{frame}

\begin{frame}
\frametitle{4) Durante uma análise de desempenho de um sistema de ventilação industrial, o engenheiro responsável precisa compreender as diferentes naturezas da pressão envolvida no escoamento do ar. Com base nos conceitos de pressão estática, pressão diferencial e pressão dinâmica, assinale a alternativa que apresenta apenas afirmações corretas.}
{\BodySize
\par I. A pressão dinâmica é exercida por um fluido em movimento e pode ser medida com uma tomada de impulso voltada contra o escoamento.; II. A pressão estática corresponde à força que o fluido exerce em repouso, sendo medida por uma tomada perpendicular ao fluxo.; III. A pressão diferencial é representada pela variação ΔP entre dois pontos distintos de um sistema.; IV. A pressão estática é medida diretamente na linha de impacto do fluido em movimento, o que garante máxima precisão.; V. A pressão dinâmica é frequentemente utilizada para medir vazão em tubos através de tubos de Pitot.; VI. A pressão diferencial só é aplicável em fluidos compressíveis, como gases industriais pressurizados.
\begin{itemize}
\item[a)] Apenas I, II e III estão corretas.
\item[b)] Apenas II, III e IV estão corretas.
\item[c)] Apenas I, IV e VI estão corretas.
\item[d)] Apenas II, V e VI estão corretas.
\item[e)] Apenas I, III e V estão corretas.
\end{itemize}
}
\end{frame}

\begin{frame}
\frametitle{4) Durante uma análise de desempenho de um sistema de ventilação industrial, o engenheiro responsável precisa compreender as diferentes naturezas da pressão envolvida no escoamento do ar. Com base nos conceitos de pressão estática, pressão diferencial e pressão dinâmica, assinale a alternativa que apresenta apenas afirmações corretas.}
{\BodySize
\par I. A pressão dinâmica é exercida por um fluido em movimento e pode ser medida com uma tomada de impulso voltada contra o escoamento.; II. A pressão estática corresponde à força que o fluido exerce em repouso, sendo medida por uma tomada perpendicular ao fluxo.; III. A pressão diferencial é representada pela variação ΔP entre dois pontos distintos de um sistema.; IV. A pressão estática é medida diretamente na linha de impacto do fluido em movimento, o que garante máxima precisão.; V. A pressão dinâmica é frequentemente utilizada para medir vazão em tubos através de tubos de Pitot.; VI. A pressão diferencial só é aplicável em fluidos compressíveis, como gases industriais pressurizados.
\begin{itemize}
\item[a)] Apenas I, II e III estão corretas.
\item[b)] Apenas II, III e IV estão corretas.
\item[c)] Apenas I, IV e VI estão corretas.
\item[d)] Apenas II, V e VI estão corretas.
\item[e)] \alert{Apenas I, III e V estão corretas.}
\end{itemize}
}
\end{frame}

\begin{frame}
\frametitle{4) Durante uma análise de desempenho de um sistema de ventilação industrial, o engenheiro responsável precisa compreender as diferentes naturezas da pressão envolvida no escoamento do ar. Com base nos conceitos de pressão estática, pressão diferencial e pressão dinâmica, assinale a alternativa que apresenta apenas afirmações corretas.}
{\BodySize
\textbf{OBS.:}
\begin{itemize}
\item Este é um exempo de observação
\item Veja como são as coisas
\end{itemize}
}
\end{frame}

\end{document}
